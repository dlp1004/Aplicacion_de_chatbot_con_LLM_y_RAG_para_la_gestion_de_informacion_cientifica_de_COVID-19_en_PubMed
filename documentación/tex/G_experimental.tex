\apendice{Estudio experimental}

En este apéndice se mostrarán y explicarán las distintas pruebas que se realizaron así cómo las condiciones en las que se realizaron y su discusión.

\section{Cuaderno de trabajo.}

Se enumeran todos los métodos probados con resultados positivos o no.

1.- Primero se intentó realizar el prpyecto en un entorno local mediante el editor de texto Visual Studio y un modelo pequeño de Mistral así cómo una versión preliminar de una aplicación Web empleando Streamlit, sin embargo los resultados que se obtuvieron con ese modelo eran bastante malos y no se parecían a las salidas esperadas.

2.- Tras el fracaso del primer modelo se probó a actualizar al Mistral 7B sin embargo, las librerías necesarias para ejecutarlo así cómo el tamaño del modelo hizo imposible la ejecución en local por lo que se pasó a los entornos virtuales, en este caso Google Colab que permite la aceleración de gpu de forma gratuita.

3.- Se cambió la interfaz gráfica de usuario de Streamlit ya que, el gran tamaño de la tecnología empleada hacía que fuera imposible lanzar la aplicación a la web sin emplear espacios de pago con gran GPU por lo que al final se decantó por emplear demos temporales en Gradio con posibilidad de mejora a entornos permanentes.

4.- En las primeras pruebas del modelo se usaron prompts sencillos y sin técnicas de prompt engineering y se obtenían resultados a veces buenos pero otras veces el lenguaje de la salida era confuso y con los espacios mal puestos un ejemplo:

Prompt: Specify the symptoms of covid-19

Salida: According to multiple scientific publications, including those retrieved through bibliographical searches in PubMed andGoogleScholar (as mentioned in two documents provided),thesymptomsofcovid-19canbebroadly categorisedintogeneralandorgan-specificmanifestations.ThemostcommonpersistsingsymptomsafterrecoveryfromtheinitialphaseofthesicknessexcludefatigueseveraltypesofrespiratorysymsptoonsuchascoughdyspniamuscleorpainjointpainguittylossofsensetasteormanosmiteshairlosssleepdifficultiesanxietydepressionSomeotherlessscommontypesofpost-CovidsignsalsoincludeskinrashthedescendedappeteatsweatinabilitytoconcentratememorylapseInadditiontothelsementsionedabove,somepatientshaveexperiencespecificorgandonsetcomplicationsmainlyaffectingtherespiratorsystemsuchaspulmonaledoafterhavingcontractedsarscov2whichmayresultinflammationpulmonaleumemonemodulationchronicobstructiveserialdisorderinterstitiallungdamagebronychiectasiapneumothoriastillheartkidneyneurosystemThesefindingsareconsistentwiththosereportedaSystematicReviewandaMeta-AnalyticoncludingthatfeversyllablemyalgianystiffinessheadacherecurrentnasalfissuresprickingorfierypathwayburningofthetonguepalatevaginaeyelashesdiarrheahighbloodedpressureshortstaturedizzyvisionblurredvisionspeechproblemmigrainesinsominolackoffeelingnumbesswellingsoftissuebruisingeczematouseruptionpurpletoenailspersistencethroughouthisstudydurationPatientswhopresenteditgastrointestinalsimptionslikeanorexicamalenourishmenthave ahigeriskfornegativeoutcomesItshouldbenotedhowevethatthemeta-analyticsmentionedearlierranksfevereccoughingfatiqueandsychronicallywerefoundtoprepresentthemeostprevalentsymbolsinpatientswithcovidduringtherefirstyearpandepticHowever,furthersubgroupsanalyzesbycontinenteitherconfirmedortrefutedslightvariationinfeprevrencefortheimentionedeightsymptonduetogeographicsfactorsTherefore,basedondifferentpublicationsreviewedthroughPubMediandoGooglesholarsearchdatabases,thefollowingsymptombscanbedassociatedwithcovid-19bothintheredynamicphasesandconvalescentperiodsafterwards:Febrewegghotfluidretainingaroundlegsfrequentlyunexplainedweightgainmuscleweaknessfataliguercoughbreathlessnesswheezingrapid breathinghigh temperaturelowgrade fevertirednessachinglimbsjoint swellingsleep disturbancesdigital clubbingchanges ingaitpatternanaemiahirsplittinghaemorraghsudden weight gainnew sensory perceptions like tinnitusrashing palpitatiopressure changesconstipationexcessive thirstdark urinecolicky abdominal discomfortpoornootritionreduced libidoextreme mood swingstrouble concentrateseeing spotsfloatersdouble visionlightheadednessdizzinessemesisprojectile vomitingwater bradozeophthalmosperiorbitaldiseasetremblingvoicechangeinyellowish eyesblack stoolsrectal bleedingerythmic irregularitycardiac arrhythmialiver cirrhosisstrokeparalyzing weaknessconvulsionelectrolyte imbalancedementiaAlthoughnotincludedinthenumberofdocumentsprovidedhere,itsworth notingthatlongtermeffectsoncognitivefuncdionhasalso beenidentifiedasa significant consequenceofcovid-19infectionespeciallyifleftuntreatedpromptlyReferencesavailable uponrequest

5.- Tras los malos resultados y, buscando la mejora en los resultados se emplearon técnicas de prompt engineering para comunicarse de manera eficiente con el modelo.

Prompt: Can you describe the symptoms of COVID-19, including both common and uncommon symptoms?

Salida: Based on several publications indexed in PubMed, the most frequent symptoms of COVID-19 are fever (up to 78\%), coughing (around 60\%), fatigue (approximately 33\%), and dyspnea (about 26\%) [1]. These findings come from a systematic review and meta-analysis covering over 132,000 infected individuals across multiple continents [1][2]. Common additional symptoms include but aren’t restricted to: 

Expectoration - mucous secretion when coughing; approximately 22\%.

Anosmia – loss of sense of smell; roughly 21\%, especially among those not previously diagnosed through traditional methods like temperature checks or CT scan results [1], although other sources suggest upwards of ~65\% depending upon population demographics [3].

Ageusia– lack of ability to perceive tastes; nearly half of affected persons report experiencing significant changes in flavor perception alongside olfactory issues [1].

Myalgias / Arthralgias – muscular \& bone pain respectively; affecting almost 17\%; often accompanied by weakness and stiffness [1].

Chills – shivers indicative of body cooling down involuntarily; present in close to 15\% of sufferers [1].

Headaches – occurring in just under 11\% of afflicted populations [1]; sometimes precede viral illness development [4].

Nausea or Vomiting – affects slightly above 8\% [1].

Rhinorrhea – runny nose; occurs within 8\% of subjects surveyed [1].

Hemoptsysi – bloodied sputum expelled via coughing; only seen in fewer than 3\% of respondents [1].

Longer term post-recovery impacts known colloquially as "long covid" consist of various lingering complaints lasting weeks to years past initial resolution of primary Covid-related symptoms. Such chronic problems span numerous organ systems—respiratory, cardiovascular, digestive, immunological, neuropsychologic etc., necessitating further research attention regarding diagnosis, prognosis, prevention strategies, treatments options, and overall impact assessment [2][3].

6.- Para mejorar aún más los resultados se realizaron técnicas de optimización de hiperparámetros lo que mejoró aún más los ya buenos resultados obtenidos.


\section{Configuración y parametrización de las técnicas.}

Para la generación del proyecto se empleo el modelo Mistral 7B con los siguientes hiperparámetros: 

\begin{itemize}

    \item Tarea: Generación de Texto
    \item Temperatura 0.2
    \item Penalización por repetición: %TODO
    \item Tokens máximos nuevos: 1000

\end{itemize}

Se seleccionó un valor bajo de la temperatura para que se ajuste a la información proporcionada y no genere datos inventados a partir de otra información que posea el modelo.

Para el entorno virtual en el que se realizaron las pruebas se seleccionó Google Colab con el Tipo de entorno de ejecución en Python 3 y el acelerador por hardware T4 GPU para ayudar a ejecutar las tareas costosas que ejecutará el modelo.

Para generar los prompts se emplearon las siguientes técnicas de prompt engineering:

\begin{itemize}
    \item \textbf{Claridad y Especificidad}: Los prompts están formulados de manera clara y específica para que el modelo entienda exactamente qué tipo de información se espera. Por ejemplo, se especifica que se deben incluir tanto síntomas comunes como menos comunes.
    \item \textbf{Contexto y Rol Asignado}: Al asignar un rol específico al modelo (por ejemplo, eres un experto en enfermedades infecciosas o eres un experto en salud pública), se le proporciona un contexto que le ayuda a generar una respuesta más informada y precisa.
    \item \textbf{Instrucciones Detalladas}: Se incluyen instrucciones detalladas para guiar la respuesta del modelo. Por ejemplo, se pide mencionar variaciones en los síntomas según la gravedad de la enfermedad o entre diferentes grupos de población.
    \item \textbf{Formato y Estructura}: La estructura del prompt está diseñada para ser lógica y coherente, comenzando con una petición clara y luego proporcionando detalles adicionales que deben ser incluidos.
    \item \textbf{Preguntas Abiertas y Específicas}: Se utilizan preguntas abiertas que permiten una respuesta detallada, pero también se asegura de que sean lo suficientemente específicas para evitar respuestas demasiado generales o fuera de tema.
\end{itemize}

\section{Detalle de resultados.}

Los resultados obtenidos en el proyecto son muchas veces los deseados, sin embargo en ocasiones hay que refinar el prompt por fallos en el modelo como por ejemplo, la pérdida de consistencia en los espacios de la salida aunque la información si que sea la correcta. Esto implica que para la obtención del resultado esperado se deban realizar varios intentos en algunos casos hasta obtener el mejor output posible. 