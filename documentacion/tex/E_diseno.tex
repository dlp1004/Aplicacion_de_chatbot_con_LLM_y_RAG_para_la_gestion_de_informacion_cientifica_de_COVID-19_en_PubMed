\apendice{Manual de especificación de diseño}



\section{Planos}

No proceden en este proyecto

\section{Diseño arquitectónico}

En esta sección se mostrarán los diagramas de secuencia de cada caso de uso lo que aportará claridad a su ejecución además de un diagrama de componentes que esquematizará los distintos miembros constituyentes presentes en el proyecto.

\subsection{Diagramas de secuencia}

\textbf{CU-1 Generar Respuesta}

\begin{figure}[h]
    \centering
    \includegraphics[width=1\textwidth]{img/dseq_CU1.png}
    \caption{Diagrama de secuencia del caso de uso número 1}
    \label{fig:dseq1}
\end{figure}

Ver imagen \ref{fig:dseq1}

\textbf{CU-2 Actualizar el modelo}

\begin{figure}[h]
    \centering
    \includegraphics[width=1\textwidth]{img/dseq_CU2.png}
    \caption{Diagrama de secuencia del caso de uso número 2}
    \label{fig:dseq2}
\end{figure}

Ver imagen \ref{fig:dseq2}

\textbf{CU-3 Modificar la información especializada}

\begin{figure}[h]
    \centering
    \includegraphics[width=1\textwidth]{img/dseq_CU3.png}
    \caption{Diagrama de secuencia del caso de uso número 3}
    \label{fig:dseq3}
\end{figure}

Ver imagen \ref{fig:dseq3}

\textbf{CU-4 Repetir el prompt}

\begin{figure}[h]
    \centering
    \includegraphics[width=1\textwidth]{img/dseq_CU4.png}
    \caption{Diagrama de secuencia del caso de uso número 4}
    \label{fig:dseq4}
\end{figure}

Ver imagen \ref{fig:dseq4}

\textbf{CU-5 Eliminar conversación}

\begin{figure}[h]
    \centering
    \includegraphics[width=1\textwidth]{img/dseq_CU5.png}
    \caption{Diagrama de secuencia del caso de uso número 5}
    \label{fig:dseq5}
\end{figure}

Ver imagen \ref{fig:dseq5}

\subsection{Diagrama de componentes}

\begin{figure}[h]
    \centering
    \includegraphics[width=1\textwidth]{img/components.png}
    \caption{Diagrama de componentes del proyecto}
    \label{fig:comps}
\end{figure}

En la figura \ref{fig:comps} se puede apreciar el diagrama de componentes del proyecto que pretende ilustrar la disposición de cada parte relevante de la aplicación así como sus interacciones. A continuación se explicará brevemente cada componente.

\begin{itemize}

    \item \textbf{Aplicación:} código general que tiene función estructural y de control del resto de componentes del proyecto.
    
    \item \textbf{HuggingFace:} comunidad de inteligencia artificial que permite el acceso mediante tokens a modelos de manera gratuita con alguna restricción.

    \item \textbf{Mistral 7B:} modelo Grande de Lenguaje (LLM) creado por Mistral, destaca por su pequeño tamaño y gran optimización en comparación a otros modelos similares lo que permite una relación coste/calidad superior.
    
    \item \textbf{GUI:} interfaz de usuario desarrollada con el framework de Gradio que permite al usuario interactuar de manera cómoda e intuitiva con el proyecto.
    
    \item \textbf{Base de datos vectorizada:} base de datos de tipo índice FAISS que contiene la información de entrenamiento para el proceso de RAG vectorizada para calcular su similitud matemática con el prompt de entrada.
    
    \item \textbf{Generación de embeddings:} proceso realizado por un modelo de machine learning que representa la información originalmente pasada en lenguaje natural como un vector medible y comparable.
    
    \item \textbf{Datos de entrenamiento:} conjunto de datos en .CSV (podrían tener otros formatos) sobre los que se quieren obtener respuestas especializadas.
    
\end{itemize}

