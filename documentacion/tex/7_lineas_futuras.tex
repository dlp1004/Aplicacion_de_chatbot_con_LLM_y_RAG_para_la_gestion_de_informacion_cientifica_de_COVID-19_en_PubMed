\capitulo{7}{Lineas de trabajo futuras}

Por su naturaleza de prueba de concepto, existen diferentes maneras en las que este trabajo se puede mejorar.

\subsection{Mejoras en la selección de Modelos}

En cuanto a los modelos se podría investigar sobre alternativas al modelo aquí empleado Mistral7B, se podrían emplear modelos más grandes y costosos computacionalmente que generen mejores resultados.

Lo mismo se podría aplicar al modelo de embeddings, se podrían investigar modelos más potentes o especializados en datos de Covid-19.

\subsection{Ampliación o acotación de los datos}

Otra mejora que podría ser interesante puede ser la ampliación del banco de datos o incluso acotarlos a un tema más específico, por ejemplo, seleccionar aquellos que hablen sobre el mecanismo de acción del virus, pudiendo incluso tokenizar artículos enteros en vez de sólo abstracts. Esto generaría un chatbot sumamente específico, en contraposición del general al que se aspira en esta prueba de concepto.

\subsection{Mejoras en la calidad gráfica}

Otra mejora que inmediatamente viene a la mente es la de generar una interfaz gráfica más pulida y detallada que aporte al proyecto un toque más profesional.

\subsection{Despliegue en la nube}

Con los espacios ofrecidos por Hugging Face %TODO insesrtar referencia a espacios de Hugging Face 
se podría desplegar en la nube el proyecto, sin embargo debido a los precios esta mejora se presenta como costosa y fuera del alcance del presupuesto de un estudiante de carrera

\subsection{Creación de un historial de mensajes}

Cómo ya ocurre con muchos chatbots se podría crear una "memoria", es decir que el modelo obtenga como información las preguntas y respuestas ya generadas.

\subsection{Creación de un sistema de traducción}

Los abstracts de documentos recuperados están en inglés por lo que si arrojas al modelo un prompt en Español, los abstracts recuperados y el prompt enriquecido con los abstracts devolverá un peor resultado al mezclar dos idiomas distintos, esto se puede arreglar aplicando una capa de traducción al prompt introducido lo que detectará en que idioma se ha introducido y, posteriormente, lo traducirá al inglés para que no haya problemas de diferencia de idiomas. 

Esto se podría hacer con otro modelo de inteligencia artificial que detecte el idioma y lo traduzca siempre al inglés para posteriormente pasarle ese resultado al modelo que enriquecerá el prompt y devolverá una salida, dicha salida será traducida de nuevo por el modelo al mismo idioma del prompt de entrada.
