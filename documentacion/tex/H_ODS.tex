\apendice{Anexo de sostenibilización curricular}

\section{Introducción}

A día de hoy, para que un proyecto cumpla con unos mínimos de calidad, no sólo vale con que la parte técnica esté correctamente desarrollada sino que es necesario que cumpla ciertas pautas de sostenibilidad para que, a parte de ser un proyecto que cumpla un propósito tecnológico también sea respetuoso con el medio ambiente y promueva el desarrollo de la sociedad promoviendo valores tan importantes como la justicia social y la economía equitativa y viable a largo plazo.

A lo largo de este apéndice se defenderá este proyecto como uno que impulsa, no sólo a mejorar la salud pública sino que también tiene cómo objetivo entender cómo situaciones como la pandemia vivida en 2020 afecta de manera distinta a diferentes grupos sociales con el fin de romper desigualdades y concienciar sobre aquellos grupos más necesitados.

\section{Competencias de sostenibilidad}

\subsection{Principio Ético}

Las guías del principio ético dictan que el proyecto "debe esforzarse por educar a la ciudadanía reconociendo el valor intrínseco de cada persona, situando la libertad y la protección de la vida como objetivos de las políticas públicas y los comportamientos individuales". Esto se puede reflejar en este proyecto ya que su ultimo fin es facilitar la labor a profesionales de la salud que luchan por la vida humana.

\subsection{Principio Holístico}

Este principio dice que "La universidad, en todas sus facetas, debe actuar desde una concepción integral e interdependiente de los componentes de la realidad social, económica y ambiental". Este proyecto busca informar del distinto impacto que la pandemia tuvo en gente que posee pocos recursos, dificultad de acceso al sistema sanitario y otras dificultados frente a gente más agraciada en ese aspecto intentando, mediante la información y divulgación, cerrar poco a poco las brechas entre ambos grupos. En el ámbito medioambiental el proyecto busca ahorrar en los grandes costes eléctricos que conllevan entrenar un modelo, reduciendo la huella energética producida sin que ello conlleve la pérdida de resultados.

\subsection{Principio de Complejidad}

El documento indica que este principio trata de "La adopción de enfoques sistémicos y transdisciplinares que permitan una mejor comprensión de la complejidad de las problemáticas sociales, económicas y ambientales, así como de la implicación en las mismas de todas las actividades ciudadanas y
profesionales". Este principio es fácil de justificar debido a la yuxtaposición de campos, tanto de la informática cómo el de la salud, empleando técnicas de ambos ámbitos para mejorar la información y la salud pública.

\subsection{Principio de Globalización}

La guía indica que "La adopción de enfoques que establezcan relaciones entre los contenidos curriculares y las realidades locales y globales".  Los abstracts recuperados en este proyecto, abarcan todo tipo de realidades y entornos afectados por esta brutal enfermedad, varios artículos hacen hincapié en los efectos que tiene el entorno social y económico a la hora de afrontar la infección.

\subsection{Principio de Transversalidad}

Se define este principio como: "integración de los contenidos dirigidos a la formación de competencias para la sostenibilidad en las diversas áreas de conocimiento, asignaturas y titulaciones" Al igual que en el principio de Complejidad, se puede ver el impacto que distintas áreas tienen sobre el proyecto por lo que, fácilmente, una persona con conocimientos en alguna de los ámbitos relacionados podría mejorar el proyecto ya sea actualizando y manteniendo la aplicación o mejorando y revisando los datos de entrenamiento. También sería de gran utilidad la ayuda de otras titulaciones, por ejemplo, en el marco legal.

\subsection{Principio de Responsabilidad Social Universitaria}

La definición que aporta la guía sobre este principio es la siguiente: "Contribución de la Universidad a la sostenibilidad de la Comunidad. Se reflejará en la gestión interna y en la colaboración con entidades y organismos en proyectos de investigación y acciones que contribuyan a mejorar la calidad de la formación universitaria y el avance en la resolución de los problemas sociales, económicos y ambientales". Este proyecto puede ayudar a estudiantes que tengan problemas a la hora de entender algún aspecto relevante del Covid-19 aún cuando los profesores estén fuera del horario lectivo. Además y debido a la flexibilidad que aportan los prompts a la hora de interactuar con el modelo la respuesta se puede adecuar a las necesidades del alumno.\cite{CRUE_2005}